\documentclass[11pt,a4paper,twocolumn]{article}

% Modernes Design und Schrift
\usepackage[utf8]{inputenc}
\usepackage[T1]{fontenc}
\usepackage[scaled]{helvet}
\renewcommand{\familydefault}{\sfdefault}
\usepackage[ngerman]{babel}

% Seitenränder – linker Rand größer für Ablage
\usepackage[left=3.5cm,right=2cm,top=2.5cm,bottom=2.5cm]{geometry}

% Tabellen und Typografie
\usepackage{booktabs}
\usepackage{microtype}
\usepackage{caption}
\captionsetup[table]{position=above}
\usepackage{siunitx} % für Maßeinheiten

% Dokumentinformationen
\title{\textbf{Strukturbeschreibung eines Einkristalls}}
% \author{Max Mustermann}
%\date{\today}

\begin{document}
    \maketitle


    A {\VAR{ crystal_colour }}, {\VAR{ crystal_shape }} shaped crystal was mounted on
    a {\VAR{ cif._diffrn_measurement_specimen_support }}
    with perfluoroether oil. {\VAR{ crystallization_method }}. Data for {\VAR{ cif.block.name }}
    were collected from a shock-cooled single crystal at \VAR{ cif._diffrn_ambient_temperature } K
    on {\VAR{ diffr_type|inv_article }}
        {\VAR{ diffr_type }} {\VAR{ diffr_device }} with {\VAR{ diffr_source|inv_article }} {\VAR{ diffr_source }}
    using a {\VAR{ monochromator }} as monochromator and {\VAR{ detector|inv_article }} {\VAR{ detector }}
    detector. The diffractometer
        BLOCK{ if lowtemp_dev }was equipped with {\VAR{ lowtemp_dev|inv_article }} {\VAR{ lowtemp_dev }}
        low temperature device and
            BLOCK{  endif }
            used {\VAR{ radiation }} radiation BLOCK{if wavelength}(λ = {\VAR{ wavelength }} Å)BLOCK{  endif }.
                All data were integrated with {\VAR{ integration_progr }} yielding {\VAR{ cif._diffrn_reflns_number }}
                reflections of which {\VAR{ cif._reflns_number_total }} where independent and
                    {\VAR{ '%0.1f'| format((cif._reflns_number_gt|float() / cif._reflns_number_total|float()) * 100) }} %
                were greater than 2σ(<i>F</i><sup>2</sup>).<sup>[{\VAR{ literature.integration|ref_num }}]</sup> A
                    {\VAR{ abstype }} absorption correction using {\VAR{ abs_details }} was applied.<sup>[{\VAR{ literature.absorption|ref_num }}]</sup>
                The structure was solved by
                    {\VAR{ solution_method }} methods with {\VAR{ solution_program }} and refined by full-matrix
                least-squares methods against
                <i>F</i><sup>2</sup>
                using {\VAR{ refinement_prog }}.&NoBreak;<sup>[{\VAR{ literature.solution|ref_num }},{\VAR{ literature.refinement|ref_num }}]</sup>
                All non-hydrogen atoms
                were refined with anisotropic displacement parameters.
                    BLOCK{ if cif.hydrogen_atoms_present }
                        {\VAR{ hydrogen_atoms }}
                        BLOCK{  endif }
                        Crystallographic data for the structures reported in this paper have been deposited
                        with the Cambridge Crystallographic Data Centre.<sup>[{\VAR{ literature.ccdc|ref_num }}]</sup> CCDC
                            BLOCK{ if cif._database_code_depnum_ccdc_archive }
                                {\VAR{ cif._database_code_depnum_ccdc_archive|replace('CCDC ', '') }}BLOCK{ else }
                                ??????
                                    BLOCK{  endif }
                                    contain the supplementary crystallographic data for this paper.
                                    These data can be obtained free of charge from The Cambridge Crystallographic Data
                                    Centre via <a href="www.ccdc.cam.ac.uk/structures">www.ccdc.cam.ac.uk/structures</a>.
                                    This report and the CIF file were generated using FinalCif.<sup>[{\VAR{ literature.finalcif|ref_num }}]</sup>
                                    \section*{Kristallstruktur}

                                    Die untersuchte Verbindung kristallisiert im monoklinen Kristallsystem. Die Kristallstruktur wurde mit Mo-K\(\alpha\)-Strahlung bei Raumtemperatur gemessen. Die wichtigsten strukturellen Parameter sind in Tabelle~\ref{tab:struktur} zusammengefasst.

                                    \begin{table}[h!]
                                        \centering
                                        \caption{Gitterparameter und Raumgruppe der untersuchten Verbindung.}
                                        \label{tab:struktur}
                                        \begin{tabular}{@{}ll@{}}
                                            \toprule
                                            \textbf{Parameter}                        & \textbf{Wert}     \\
                                            \midrule
                                            Raumgruppe                                & $P2_1/c$ (Nr. 14) \\
                                            a / \si{\angstrom}                        & 7.123(2)          \\
                                            b / \si{\angstrom}                        & 12.456(3)         \\
                                            c / \si{\angstrom}                        & 9.789(2)          \\
                                            $\beta$ / \si{\degree}                    & 101.23(1)         \\
                                            Volumen / \si{\angstrom^3}                & 842.5(5)          \\
                                            Z                                         & 4                 \\
                                            Dichte / \si{\gram\per\cubic\centi\metre} & 2.36              \\
                                            \bottomrule
                                        \end{tabular}
                                    \end{table}

                                    \section*{Diskussion}

                                    Die Raumgruppe $P2_1/c$ ist eine der häufigsten monoklinen Raumgruppen und erlaubt eine dichte Packung der Moleküle. Die Werte der Gitterparameter sind typisch für kleine bis mittelgroße anorganische Molekülverbindungen.

                                    \section*{Fazit}

                                    Die Strukturaufklärung zeigt, dass die Verbindung eine stabile monokline Kristallstruktur besitzt. Die Analyse der atomaren Positionen liefert die Grundlage für weiterführende Untersuchungen, etwa zur elektronischen Struktur oder zum Bindungsverhalten.


\end{document}

\end{document}