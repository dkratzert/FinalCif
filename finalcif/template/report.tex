\documentclass[11pt,a4paper,twocolumn]{article}

\usepackage[utf8]{inputenc}
\usepackage[T1]{fontenc}
\usepackage[scaled]{helvet}
\renewcommand{\familydefault}{\sfdefault}
\usepackage[ngerman]{babel}

\usepackage[left=3.5cm,right=2cm,top=2.5cm,bottom=2.5cm]{geometry}

\usepackage{booktabs}
\usepackage{microtype}
\usepackage{caption}
\captionsetup[table]{position=above}
\usepackage{siunitx}
\usepackage{verbatim}
%\usepackage{unicode-math}
\usepackage{amsmath}

\title{\textbf{  Crystal Structure Report for \detokenize{\VAR{ cif.block.name }} }}
\author{}
\date{}

\begin{document}
    \maketitle


    A
    \BLOCK{ if crystal_colour }
        \VAR{ crystal_colour },
    \BLOCK{ endif }
    \BLOCK{ if crystal_shape }
        \VAR{ crystal_shape } shaped
    \BLOCK{ endif }
    crystal was mounted on
    \BLOCK{ if cif._diffrn_measurement_specimen_support }
        a \VAR{ cif._diffrn_measurement_specimen_support }
        with perfluoroether oil.
    \BLOCK{ else }
        the goniometer.
    \BLOCK{ endif }
    \VAR{ crystallization_method }. Data for \detokenize{\VAR{ cif.block.name }}
    were collected from a shock-cooled single crystal at \VAR{ cif._diffrn_ambient_temperature }~K
    on \VAR{ diffr_type|inv_article } \VAR{ diffr_type } \VAR{ diffr_device } with \VAR{ diffr_source|inv_article } \VAR{ diffr_source }
    using a \VAR{ monochromator } as monochromator and \VAR{ detector|inv_article } \VAR{ detector }
    detector. The diffractometer
    \BLOCK{ if lowtemp_dev }
        was equipped with \VAR{ lowtemp_dev|inv_article } \VAR{ lowtemp_dev } low temperature device and
    \BLOCK{  endif }
    used \VAR{ radiation } radiation \BLOCK{if wavelength} ($\lambda$ = \VAR{ wavelength }~\AA)\BLOCK{ endif }.
    All data were integrated with \VAR{ integration_progr } yielding \VAR{ cif._diffrn_reflns_number }
    reflections of which \VAR{ cif._reflns_number_total } where independent and
        \VAR{ '%0.1f'| format((cif._reflns_number_gt|float() / cif._reflns_number_total|float()) * 100) } \%
    were greater than 2$\sigma$(\textit{F}$^2$).$^{[\VAR{ literature.integration|ref_num }]}$ A
        \VAR{ abstype } absorption correction using \VAR{ abs_details } was applied.$^{[\VAR{ literature.absorption|ref_num }]}$
    The structure was solved by
        \VAR{ solution_method } methods with \VAR{ solution_program } and refined by full-matrix
    least-squares methods against
    \textit{F}$^{2}$
    using \VAR{ refinement_prog }.\nobreak$^{[\VAR{ literature.solution|ref_num },\VAR{ literature.refinement|ref_num }]}$
    All non-hydrogen atoms were refined with anisotropic displacement parameters.
    \BLOCK{ if cif.hydrogen_atoms_present }
        \VAR{ hydrogen_atoms }
    \BLOCK{  endif }
    Crystallographic data for the structures reported in this paper have been deposited
    with the Cambridge Crystallographic Data Centre.$^{[\VAR{ literature.ccdc|ref_num }]}$ CCDC
    \BLOCK{ if cif._database_code_depnum_ccdc_archive }
    \VAR{ cif._database_code_depnum_ccdc_archive|replace('CCDC ', '') }\BLOCK{ else }
        ??????
    \BLOCK{ endif }
    contain the supplementary crystallographic data for this paper.
    These data can be obtained free of charge from The Cambridge Crystallographic Data
    Centre via \textit{www.ccdc.cam.ac.uk/structures}.
    This report and the CIF file were generated using FinalCif.$^{[\VAR{ literature.finalcif|ref_num }]}$

    \BLOCK{ if refinement_details }
        \section*{Refinement details for \detokenize{\VAR{ cif.block.name }}}
            \VAR{ refinement_details }
    \BLOCK{ endif }

    \pagebreak
    
    \section*{Table 1. Crystal data and structure refinement for \detokenize{\VAR{ cif.block.name }}}

\end{document}