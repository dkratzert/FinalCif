\documentclass[10pt,a4paper,twocolumn]{article}

\usepackage[utf8]{inputenc}
\usepackage[T1]{fontenc}
\usepackage[scaled]{helvet}
\renewcommand{\familydefault}{\sfdefault}
\usepackage[ngerman]{babel}

\usepackage[left=1.5cm,right=1cm,top=2.0cm,bottom=2.0cm]{geometry}

\usepackage{booktabs}
\usepackage{microtype}
\usepackage{caption}
\captionsetup[table]{position=above}
\usepackage{siunitx}
\usepackage{verbatim}
%\usepackage{unicode-math}
\usepackage{amsmath}
\usepackage{upgreek}

\title{\textbf{  Crystal Structure Report for \detokenize{\VAR{ cif.block.name }} }}
\author{}
\date{}

\begin{document}
    \maketitle


    A
    \BLOCK{ if crystal_colour }
        \VAR{ crystal_colour },
    \BLOCK{ endif }
    \BLOCK{ if crystal_shape }
        \VAR{ crystal_shape } shaped
    \BLOCK{ endif }
    crystal was mounted on
    \BLOCK{ if cif._diffrn_measurement_specimen_support }
        a \VAR{ cif._diffrn_measurement_specimen_support }
        with perfluoroether oil.
    \BLOCK{ else }
        the goniometer.
    \BLOCK{ endif }
    \VAR{ crystallization_method }. Data for \detokenize{\VAR{ cif.block.name }}
    were collected from a shock-cooled single crystal at \VAR{ cif._diffrn_ambient_temperature }~K
    on \VAR{ diffr_type|inv_article } \VAR{ diffr_type } \VAR{ diffr_device } with \VAR{ diffr_source|inv_article } \VAR{ diffr_source }
    using a \VAR{ monochromator } as monochromator and \VAR{ detector|inv_article } \VAR{ detector }
    detector. The diffractometer
    \BLOCK{ if lowtemp_dev }
        was equipped with \VAR{ lowtemp_dev|inv_article } \VAR{ lowtemp_dev } low temperature device and
    \BLOCK{  endif }
    used \VAR{ radiation } radiation \BLOCK{if wavelength} ($\lambda$ = \VAR{ wavelength }~\AA)\BLOCK{ endif }.
    All data were integrated with \VAR{ integration_progr } yielding \VAR{ cif._diffrn_reflns_number }
    reflections of which \VAR{ cif._reflns_number_total } where independent and
        \VAR{ '%0.1f'| format((cif._reflns_number_gt|float() / cif._reflns_number_total|float()) * 100) } \%
    were greater than 2$\sigma$(\textit{F}$^2$).$^{[\VAR{ literature.integration|ref_num }]}$ A
        \VAR{ abstype } absorption correction using \VAR{ abs_details } was applied.$^{[\VAR{ literature.absorption|ref_num }]}$
    The structure was solved by
        \VAR{ solution_method } methods with \VAR{ solution_program } and refined by full-matrix
    least-squares methods against
    \textit{F}$^{2}$
    using \VAR{ refinement_prog }.\nobreak$^{[\VAR{ literature.solution|ref_num },\VAR{ literature.refinement|ref_num }]}$
    All non-hydrogen atoms were refined with anisotropic displacement parameters.
    \BLOCK{ if cif.hydrogen_atoms_present }
        \VAR{ hydrogen_atoms }
    \BLOCK{  endif }
    Crystallographic data for the structures reported in this paper have been deposited
    with the Cambridge Crystallographic Data Centre.$^{[\VAR{ literature.ccdc|ref_num }]}$ CCDC
    \BLOCK{ if cif._database_code_depnum_ccdc_archive }
    \VAR{ cif._database_code_depnum_ccdc_archive|replace('CCDC ', '') }\BLOCK{ else }
        ??????
    \BLOCK{ endif }
    contain the supplementary crystallographic data for this paper.
    These data can be obtained free of charge from The Cambridge Crystallographic Data
    Centre via \textit{www.ccdc.cam.ac.uk/structures}.
    This report and the CIF file were generated using FinalCif.$^{[\VAR{ literature.finalcif|ref_num }]}$

    \BLOCK{ if refinement_details }
        \section*{Refinement details for \detokenize{\VAR{ cif.block.name }}}
            \VAR{ refinement_details }
    \BLOCK{ endif }

    \pagebreak
    
    \section*{Table 1. Crystal data and structure refinement for \detokenize{\VAR{ cif.block.name }}}

    \begin{table}[]
    \begin{tabular}{ll}
    CCDC number                         & \VAR{cif._database_code_depnum_ccdc_archive}  \\
    Empirical formula                   &  \VAR{sum_formula} \\
    Formula weight                      & \VAR{ cif._chemical_formula_weight } \\
    Temperature [K]                     & \VAR{ cif._diffrn_ambient_temperature } \\
    Crystal system                      &  \VAR{cif._space_group_crystal_system} \\
    Space group (number)                &  \VAR{space_group}  \\
    \textit{a} [Å]                      &  \VAR{cif._cell_length_a}  \\
    \textit{b} [Å]                      &  \VAR{cif._cell_length_b}  \\
    \textit{c} [Å]                      &  \VAR{cif._cell_length_c}  \\
    \textit{alpha} [°]                  &  \VAR{cif._cell_angle_alpha}  \\
    \textit{beta} [°]                   &  \VAR{cif._cell_angle_beta}  \\
    \textit{gamma} [°]                  &  \VAR{cif._cell_angle_gamma}  \\
    Volume [\AA$^3$]                    &  \VAR{cif._cell_volume}  \\
    \textit{Z}                          &  \VAR{cif._cell_formula_units_Z}  \\
    $\rho_{calc}$ [gcm$^{-3}$]          &  \VAR{cif._exptl_crystal_density_diffrn}  \\
    $\mu$ [mm$^{-1}$]                   &  \VAR{cif._exptl_absorpt_coefficient_mu}  \\
    $F$(000)                            &  \VAR{cif._exptl_crystal_F_000}  \\
    Crystal size [mm$^3$]               &  \VAR{crystal_size}  \\
    Crystal colour                      &  \VAR{crystal_colour}  \\
    Crystal shape                       &  \VAR{crystal_shape}  \\
    Radiation                           &  \VAR{radiation} \BLOCK{if wavelength} ($\lambda$=\VAR{ wavelength }\nobreakspace\AA)\BLOCK{ endif } \\
    2$\uptheta$ range [°]               &  \VAR{theta_range}  \\
    Index ranges                        &  \VAR{index_ranges}  \\
    Reflections collected               &  \VAR{cif._diffrn_reflns_number}  \\
    Independent reflections             &  \VAR{indepentent_refl}  \\
                                        &  Rint = \VAR{r_int}            \\
                                        &  Rsigma = \VAR{r_sigma}        \\
    Completeness
    \BLOCK{ if theta_full } to
      $\uptheta$ = \VAR{ theta_full }°
    \BLOCK{ endif }                     &  \VAR{completeness}\% \\
    Data / Restraints / Parameters      &  \VAR{ data } / \VAR{ restraints } / \VAR{ parameters }  \\
    Absorption correction Tmin/Tmax     &  \VAR{ t_min } / \VAR{ t_max } \BLOCK{ if abstype }(\VAR{ abstype })\BLOCK{ endif }  \\
    Goodness-of-fit on $F^2$            &  \VAR{goof}  \\
    Final R indexes                     &  $R_1$ = \VAR{ ls_R_factor_gt }  \\
    \([I \geq 2\sigma(I)]\)             &  $wR_2$ = \VAR{ ls_wR_factor_gt }      \\
    Final R Indexes                     &  $R_1$ = \VAR{ ls_R_factor_all }  \\
    {[all data]}                        &  $wR_2$ = \VAR{ ls_wR_factor_ref }  \\
    Largest peak/hole {[e\AA$^{-3}$]}   &  \VAR{ diff_dens_max }/\VAR{ diff_dens_min }  \\
    \BLOCK{ if exti }
      Extinction coefficient            &  \VAR{ exti }    \\
    \BLOCK{ endif }
    \BLOCK{ if flack_x }
      Flack X parameter                 &  \VAR{flack_x}  \\
    \BLOCK{ endif }
    \end{tabular}
    \end{table}

    \BLOCK{ if atomic_coordinates }
        \section*{Table 2. Atomic coordinates and \textit{U}$_{eq}$ {[\AA$^2$]} for \detokenize{\VAR{ cif.block.name }}}

 %   <caption><i>U</i><sub>eq</sub> is defined as 1/3 of the trace of the orthogonalized<i>U</i><sub>ij</sub> tensor.</caption>

        \begin{table}[]
        \begin{tabular}{lllll}
            \BLOCK{ for atom in atomic_coordinates }
                \VAR{ atom.label } & \VAR{ atom.x|replace('\u2212', '\\textminus') } &
                \VAR{ atom.y|replace('\u2212', '\\textminus') } & \VAR{ atom.z|replace('\u2212', '\\textminus') } &
                \VAR{ atom.u_eq }  \\
            \BLOCK{ endfor }
        \end{tabular}
        \end{table}
    \BLOCK{ endif }

    \BLOCK{ if displacement_parameters and options.report_adp }
        %Table 3. Anisotropic displacement parameters [Å<sup>2</sup>] for {{ cif.block.name }}.
        %        The anisotropic displacement factor exponent takes the form: −2π<sup>2</sup>[&thinsp;<i>h</i><sup>2</sup>(<i>a</i><sup>*</sup>)<sup>2</sup><i>U</i><sub>11</sub>&thinsp;+&thinsp;<i>k</i><sup>2</sup>(<i>b</i><sup>*</sup>)<sup>2</sup><i>U</i><sub>22</sub>&thinsp;+&thinsp;…&thinsp;+&thinsp;2<i>hka</i><sup>*</sup><i>b</i><sup>*</sup><i>U</i><sub>12</sub>&thinsp;]

    % <caption><i>U</i><sub>eq</sub> is defined as 1/3 of the trace of the orthogonalized <i>U</i><sub>ij</sub> tensor.

        \begin{table}[]
        \begin{tabular}{lllllll}
           % <th>Atom  <i>U</i><sub>11</sub>  <i>U</i><sub>22</sub>   <i>U</i><sub>33</sub>   <i>U</i><sub>23</sub>   <i>U</i><sub>13</sub>   <i>U</i><sub>12</sub>
            \BLOCK{ for atom in displacement_parameters }
                \VAR{ atom.label }  &  \VAR{ atom.U11|replace('\u2212', '\\textminus') }  &  \VAR{ atom.U22|replace('\u2212', '\\textminus') }
            &  \VAR{ atom.U33|replace('\u2212', '\\textminus') } &  \VAR{ atom.U23|replace('\u2212', '\\textminus') }
            &  \VAR{ atom.U13|replace('\u2212', '\\textminus') }  &  \VAR{ atom.U12|replace('\u2212', '\\textminus') }  \\
            \BLOCK{ endfor }
        \end{tabular}
        \end{table}

    \BLOCK{ endif }


\end{document}